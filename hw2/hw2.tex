\documentclass[9pt]{article}
\usepackage{amsmath} 
\usepackage{amsfonts}

% 加载宏包
\usepackage[utf8]{inputenc} 
\usepackage[T1]{fontenc}   
\usepackage{textcomp}      
\usepackage{CJKutf8}        

\title{Hw2}           
\author{Nmae: Wei Jun Li \\ Rin: 662006326}   
\date{\today}               

\begin{document}
\begin{CJK*}{UTF8}{gbsn}    

\maketitle                  

% 文档的主体内容开始
\section{problem 1}
\subsection{ 85871 mod 67}
    formual: $a b = ((a \% x)(b \% x) \% x)$

    So we have:
    
    85871 \% 67 == ((43\% 67) (1997 \%67)\%67)

\hspace*{2 cm}  == ((43 * 53)\% 67)

\hspace*{2 cm}  == 2322 \% 67

\hspace*{2 cm}  == 44

\subsection{ 5 mod -11}
    formual: $a = qn + b $

    So we have:
    
    5 = -11 * q + b

    q = -1

    so we get the result: 5 == 11 + b => b = -6
    
    The anwer we get is  5 mod -11 is equal to -6

\subsection{-149 mod 19}
    formual: $a = qn + b $

    So we have:
    
    -149 = 19 * q + b

    q = -8

    so we get the result: -149 == -152 + b => b = 3
    
    The anwer we get is  -149 mod 19 is equal to 3

\newpage

\section{problem 2}

\subsection{ $17^{30} \mod  31$}

We aim to calculate \(17^{30} \mod 31\). By Fermat's Little Theorem, if \(p\) is a prime number, and \(a\) is a positive integer less than \(p\) and co-prime to \(p\), then \(a^{p-1} \equiv 1 \mod p\).

Since \(31\) is a prime number and \(17\) is less than \(31\), we can assert:
\begin{equation*}
17^{30} \equiv 1 \mod 31
\end{equation*}

This is because:
\begin{align*}
gcd(17, 31) &= 1 \quad \text{(17 and 31 are co-prime)} \\
17^{31-1} &= 17^{30} \\
\hspace*{6 cm} 17^{30} \mod 31 &= 1
\end{align*}

Therefore, we can conclude that \(17^{30} \mod 31 = 1\) directly, without the need for extensive calculation.

\subsection{ $53^{1069} \mod 54$}

As we know that:
\begin{align*}
53 \equiv &-1 \mod 54 \\
53^{1069} \mod 54 &= (-1)^{1069} \mod 54 \\
&= -1 \mod 54 \\
&= 53
\end{align*}

Furthermore, for any odd exponent \( n \):

\begin{align*}
(-1)^n \mod d &= (-1) \mod d \\
53^{1069} \mod 54 &= (-1)^{1069} \mod 54 \\
&= (-1) \cdot 1 \mod 54 \\
&= 53^{1} \mod 54 \\
&= 53
\end{align*}

Therefore, \(53^{1069} \mod 54 = 53\).

\newpage

\section{problem 3}

\subsection{$ 13^{-1} \mod 101$}

To find the modular inverse of \(13\) modulo \(101\), we seek an integer \(x\) such that \(13x \equiv 1 \mod 101\). This problem is equivalent to finding \(x\) and \(y\) that satisfy \(13x + 101y = 1\), which can be solved using the Extended Euclidean Algorithm.

\subsection*{Extended Euclidean Algorithm}
The algorithm finds integers \(x\) and \(y\) such that \(ax + by = \gcd(a, b)\). For our case, \(a = 13\) and \(b = 101\), and we apply the algorithm to compute \(x\) in \(13x + 101y = \gcd(13, 101)\).

\subsection*{Solution}
By applying the Extended Euclidean Algorithm, we initially obtain \(x = -31\), which is the coefficient of \(13\) in the Bézout's identity. However, since we require a positive solution within the modular system, we calculate \(x \mod 101\), yielding \(x = 70\).

Therefore, the modular inverse of \(13\) modulo \(101\) is \(70\), which can be formally written as \(13^{-1} \mod 101 = 70\).


\subsection{$1234^{-1} \mod 4321$}

To find the modular inverse of \(1234\) modulo \(4321\), we need to solve for \(x\) in the congruence \(1234x \equiv 1 \mod 4321\). This can be transformed into finding \(x\) and \(y\) that satisfy \(1234x + 4321y = 1\), which is a linear Diophantine equation. The Extended Euclidean Algorithm provides a way to find such \(x\) and \(y\).

\subsection*{Extended Euclidean Algorithm}
The algorithm is based on the principle that \(\gcd(a, b)\) can be expressed as \(ax + by\), where \(x\) and \(y\) are integers. For our case, we apply the algorithm to find \(x\) in the equation \(1234x + 4321y = \gcd(1234, 4321)\).

Initially, we compute \(\gcd(1234, 4321)\), which is \(1\), indicating that \(1234\) and \(4321\) are coprime and an inverse exists. The steps are as follows:

\begin{enumerate}
    \item Apply the algorithm recursively until \(a = 0\). In our base case, we return \(b = \gcd(a, b)\), \(x = 0\), and \(y = 1\).
    \item For each recursive step, calculate \(b \mod a\), and update \(x\) and \(y\) based on the recursion: \(x = y - \left\lfloor \frac{b}{a} \right\rfloor \cdot x\), \(y = x\).
\end{enumerate}

\subsection*{Solution}
Through the Extended Euclidean Algorithm, we find that the modular inverse of \(1234\) modulo \(4321\) is \(x\), where \(x\) is adjusted to be positive by taking \(x \mod 4321\).

\subsubsection*{Calculation}
The specific calculation yields \(x = -1082\), but since we require a positive integer in the range of \(0\) to \(4320\) (inclusive), we adjust \(x\) by computing \(x \mod 4321\), resulting in \(x = 3239\).

Therefore, the modular inverse of \(1234\) modulo \(4321\) is \(3239\).

\newpage

\section{Problem 4}

\subsection*{4.1 $(\frac{x^3 + 1}{x + 1}) \mod (x^3 + x^2 + 1)$}

Given a polynomial division in \(GF(2^n)\), we aim to calculate $(\frac{x^3 + 1}{x + 1})$ and then find its modulo over $(x^3 + x^2 + 1)$. In \(GF(2^n)\), polynomial arithmetic follows unique rules where addition is equivalent to the XOR operation, and multiplication follows polynomial multiplication rules modulo a reducing polynomial, which, in this case, is not required since the division and modulo operation do not increase the polynomial degree.

\subsubsection*{Calculation}

\begin{align*}
\frac{x^3 + 1}{x + 1} &= x^2 + x + 1
\end{align*}

As per the arithmetic in \(GF(2^n)\), the division yields a polynomial of $x^2 + x + 1$, which is already in its simplest form and does not require further reduction by the modulus $(x^3 + x^2 + 1)$.

Therefore, the result of $(\frac{x^3 + 1}{x + 1} \mod (x^3 + x^2 + 1)) = x^2 + x + 1$.\\

\subsection*{4.2 $(x^8 + x^4 + x^2 + x + 1) \mod (x^6 + x + 1)$}

Given the polynomial $f(x) = x^8 + x^4 + x^2 + x + 1$ and the modulus $g(x) = x^6 + x + 1$ in \(GF(2^n)\), we aim to find $f(x) \mod g(x)$.

\subsubsection*{Polynomial Division}

To perform the division $f(x) \div g(x)$, we need to align the highest degree term of $g(x)$ with that of $f(x)$ by multiplying $g(x)$ by an appropriate monomial. The process involves multiple steps, where in each step, we subtract (in \(GF(2^n)\), subtraction is the same as addition) the product from $f(x)$ to get the remainder. This process is repeated until the degree of the remainder is less than the degree of $g(x)$.

\subsubsection*{Steps}

\begin{enumerate}
    \item Multiply $g(x)$ by $x^2$ to match the highest degree term of $f(x)$, resulting in $x^8 + x^3 + x^2$. Subtract this from $f(x)$ to get the new remainder $r_1(x) = x^4 + x^3 + x + 1$.
    \item For the next step, notice that the highest degree term of the new remainder $r_1(x)$ is $x^4$, which is lower than the degree of $g(x)$, thus stopping the division process.
\end{enumerate}

\subsubsection*{Result}

Therefore, the remainder of $f(x) \div g(x)$ in \(GF(2^n)\) is $r_1(x) = x^4 + x^3 + x + 1$, which means $f(x) \mod g(x) = x^4 + x^3 + x + 1$.

\newpage


\section{Problem 5}
Use Fermat's theorem to find:


\subsection*{5.1 $3^{201} \mod 11$}


Fermat's Little Theorem states that if \(p\) is a prime number and \(a\) is an integer not divisible by \(p\), then \(a^{p-1} \equiv 1 \mod p\). This theorem can be used to compute large exponents modulo a prime number efficiently.

\subsection*{Calculation of \(3^{201} \mod 11\)}

Given that \(11\) is a prime number and using Fermat's Little Theorem, we know that:

\begin{align*}
3^{10} &\equiv 1 \mod 11
\end{align*}

For any integer \(a\), and \(k\) being a multiple of \(10\), the exponent can be broken down as:

\begin{align*}
a^{k + 1} &= a^k \cdot a \\
&= (a^{10})^{\frac{k}{10}} \cdot a
\end{align*}

Applying this to our case with \(a = 3\) and \(k = 200\), we get:

\begin{align*}
3^{201} &= (3^{10})^{20} \cdot 3 \\
&\equiv 1^{20} \cdot 3 \mod 11 \\
&\equiv 3 \mod 11
\end{align*}

Thus, by Fermat's Little Theorem, we conclude that:

\begin{align*}
3^{201} \mod 11 &= 3
\end{align*}

\subsection*{5.2 a number x between 0 and 28 where $x^{85} \mod 29=6.$}
We want to find a number \( x \) between \( 0 \) and \( 28 \) such that \( x^{85} \mod 29 = 6 \). Fermat's Little Theorem tells us that if \( p \) is a prime number and \( a \) is an integer that is not divisible by \( p \), then \( a^{p-1} \equiv 1 \mod p \).

Since \( 29 \) is a prime number, by Fermat's Little Theorem, for any \( x \) not divisible by \( 29 \):

\begin{align*}
x^{28} &\equiv 1 \mod 29
\end{align*}

The exponent \( 85 \) can be written as \( 3 \times 28 + 1 \), so \( x^{85} \) can be expressed as:

\begin{align*}
x^{85} &= x^{84} \times x \\
&= (x^{28})^3 \times x \\
&\equiv 1^3 \times x \mod 29 \\
&\equiv x \mod 29
\end{align*}

Given that \( x^{85} \equiv 6 \mod 29 \), we can deduce that:

\begin{align*}
x &\equiv 6 \mod 29
\end{align*}

\subsection*{Conclusion}

Therefore, the number \( x \) that satisfies \( x^{85} \mod 29 = 6 \) is \( 6 \), which is within the range from \( 0 \) to \( 28 \).

\newpage

\section*{In a public-key system using RSA, you intercept the ciphertext C = 10 sent to a user whose public key is e = 5, n = 35. What is the plaintext M?}
Given the RSA encryption where the ciphertext \( C \) is \( 10 \), the public key exponent \( e \) is \( 5 \), and the modulus \( n \) is \( 35 \), the goal is to find the plaintext message \( M \).

\subsection*{RSA Decryption}

The RSA decryption requires finding the private key \( d \), which is the modular multiplicative inverse of \( e \) modulo \( \phi(n) \), where \( \phi(n) \) is the Euler's totient function of \( n \).

Since \( n = 35 \) and it is the product of two primes \( p = 5 \) and \( q = 7 \), we have:

\begin{align*}
\phi(n) &= (p-1)(q-1) \\
&= (5-1)(7-1) \\
&= 4 \cdot 6 \\
&= 24
\end{align*}

The private key \( d \) satisfies the congruence:

\begin{align*}
ed &\equiv 1 \mod \phi(n) \\
5d &\equiv 1 \mod 24
\end{align*}

Using the modular inverse function, we find that \( d = 5 \).

The plaintext message \( M \) is then found using the following congruence:

\begin{align*}
M &\equiv C^d \mod n \\
&\equiv 10^5 \mod 35
\end{align*}

Upon calculation, we determine that the plaintext message \( M = 5 \).

\end{CJK*}
\end{document}
